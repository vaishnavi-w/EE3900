\documentclass[journal,12pt,twocolumn]{IEEEtran}

\usepackage{setspace}
\usepackage{gensymb}

\singlespacing


\usepackage[cmex10]{amsmath}

\usepackage{amsthm}

\usepackage{mathrsfs}
\usepackage{txfonts}
\usepackage{stfloats}
\usepackage{bm}
\usepackage{cite}
\usepackage{cases}
\usepackage{subfig}

\usepackage{longtable}
\usepackage{multirow}

\usepackage{enumitem}
\usepackage{mathtools}
\usepackage{steinmetz}
\usepackage{tikz}
\usepackage{circuitikz}
\usepackage{verbatim}
\usepackage{tfrupee}
\usepackage[breaklinks=true]{hyperref}
\usepackage{graphicx}
\usepackage{tkz-euclide}
\usepackage{float}

\usetikzlibrary{calc,math}
\usepackage{listings}
    \usepackage{color}                                            %%
    \usepackage{array}                                            %%
    \usepackage{longtable}                                        %%
    \usepackage{calc}                                             %%
    \usepackage{multirow}                                         %%
    \usepackage{hhline}                                           %%
    \usepackage{ifthen}                                           %%
    \usepackage{lscape}     
\usepackage{multicol}
\usepackage{chngcntr}

\DeclareMathOperator*{\Res}{Res}

\renewcommand\thesection{\arabic{section}}
\renewcommand\thesubsection{\thesection.\arabic{subsection}}
\renewcommand\thesubsubsection{\thesubsection.\arabic{subsubsection}}

\renewcommand\thesectiondis{\arabic{section}}
\renewcommand\thesubsectiondis{\thesectiondis.\arabic{subsection}}
\renewcommand\thesubsubsectiondis{\thesubsectiondis.\arabic{subsubsection}}


\hyphenation{op-tical net-works semi-conduc-tor}
\def\inputGnumericTable{}                                 %%

\lstset{
%language=C,
frame=single, 
breaklines=true,
columns=fullflexible
}
\begin{document}
\newtheorem{theorem}{Theorem}[section]
\newtheorem{problem}{Problem}
\newtheorem{proposition}{Proposition}[section]
\newtheorem{lemma}{Lemma}[section]
\newtheorem{corollary}[theorem]{Corollary}
\newtheorem{example}{Example}[section]
\newtheorem{definition}[problem]{Definition}

\newcommand{\BEQA}{\begin{eqnarray}}
\newcommand{\EEQA}{\end{eqnarray}}
\newcommand{\define}{\stackrel{\triangle}{=}}
\bibliographystyle{IEEEtran}
\providecommand{\mbf}{\mathbf}
\providecommand{\pr}[1]{\ensuremath{\Pr\left(#1\right)}}
\providecommand{\qfunc}[1]{\ensuremath{Q\left(#1\right)}}
\providecommand{\sbrak}[1]{\ensuremath{{}\left[#1\right]}}
\providecommand{\lsbrak}[1]{\ensuremath{{}\left[#1\right.}}
\providecommand{\rsbrak}[1]{\ensuremath{{}\left.#1\right]}}
\providecommand{\brak}[1]{\ensuremath{\left(#1\right)}}
\providecommand{\lbrak}[1]{\ensuremath{\left(#1\right.}}
\providecommand{\rbrak}[1]{\ensuremath{\left.#1\right)}}
\providecommand{\cbrak}[1]{\ensuremath{\left\{#1\right\}}}
\providecommand{\lcbrak}[1]{\ensuremath{\left\{#1\right.}}
\providecommand{\rcbrak}[1]{\ensuremath{\left.#1\right\}}}
\theoremstyle{remark}
\newtheorem{rem}{Remark}
\newcommand{\sgn}{\mathop{\mathrm{sgn}}}
\providecommand{\abs}[1]{\vert#1\vert}
\providecommand{\res}[1]{\Res\displaylimits_{#1}} 
\providecommand{\norm}[1]{\lVert#1\rVert}
%\providecommand{\norm}[1]{\lVert#1\rVert}
\providecommand{\mtx}[1]{\mathbf{#1}}
\providecommand{\mean}[1]{E[ #1 ]}
\providecommand{\fourier}{\overset{\mathcal{F}}{ \rightleftharpoons}}
%\providecommand{\hilbert}{\overset{\mathcal{H}}{ \rightleftharpoons}}
\providecommand{\system}{\overset{\mathcal{H}}{ \longleftrightarrow}}
	%\newcommand{\solution}[2]{\textbf{Solution:}{#1}}
\newcommand{\solution}{\noindent \textbf{Solution: }}
\newcommand{\cosec}{\,\text{cosec}\,}
\providecommand{\dec}[2]{\ensuremath{\overset{#1}{\underset{#2}{\gtrless}}}}
\newcommand{\myvec}[1]{\ensuremath{\begin{pmatrix}#1\end{pmatrix}}}
\newcommand{\mydet}[1]{\ensuremath{\begin{vmatrix}#1\end{vmatrix}}}
\numberwithin{equation}{subsection}
\makeatletter
\@addtoreset{figure}{problem}
\makeatother
\let\StandardTheFigure\thefigure
\let\vec\mathbf
\renewcommand{\thefigure}{\theproblem}
\def\putbox#1#2#3{\makebox[0in][l]{\makebox[#1][l]{}\raisebox{\baselineskip}[0in][0in]{\raisebox{#2}[0in][0in]{#3}}}}
     \def\rightbox#1{\makebox[0in][r]{#1}}
     \def\centbox#1{\makebox[0in]{#1}}
     \def\topbox#1{\raisebox{-\baselineskip}[0in][0in]{#1}}
     \def\midbox#1{\raisebox{-0.5\baselineskip}[0in][0in]{#1}}
\vspace{3cm}
\title{EE3900-Assignment 3}
\author{W Vaishnavi\\AI20BTECH11025}
\maketitle
\newpage
\bigskip
\renewcommand{\thefigure}{\theenumi}
\renewcommand{\thetable}{\theenumi}
Download all latex-tikz codes from 
%
\begin{lstlisting}
https://github.com/vaishnavi-w/EE3900/blob/main/Assignment3/latex3.tex
\end{lstlisting}
and python codes from 
%
\begin{lstlisting}
https://github.com/vaishnavi-w/EE3900/blob/main/Assignment3/codes
\end{lstlisting}
\section{Ramsey 4.4 Systems of circles Q.3}
Find the equation of a circle which cuts orthogonally the two circles 
\begin{align}
    S_1 = \vec{x}^\top\vec{x} - \myvec{2&2}\vec{x} + 1 = 0 \label{circle1}\\
    S_2 = \vec{x}^\top\vec{x} + \myvec{-3&6}\vec{x} - 2 = 0 \label{circle2}
\end{align}
and passes through the point \myvec{-3\\2}
\section{Solution}
\begin{lemma}
Orthogonality of circles : Two circles are said to be orthogonal if they meet at right angles i.e the tangents at their points of intersection are perpendicular to each other.
\begin{figure}[h!]
\centering
\includegraphics[width=\columnwidth]{orthogonal.png}
\label{fig:orthogonality1}
\caption{Orthogonal circles}
\end{figure}
Given two circles,
\begin{align}
    \vec{x}^\top\vec{x} - 2\vec{c_1}^\top\vec{x} + f_1 = 0\\
    \vec{x}^\top\vec{x} - 2\vec{c_2}^\top\vec{x} + f_2 = 0
\end{align}
They are orthogonal if
\begin{align}
    2\vec{c_1}^\top\vec{c_2} = f_1 + f_2 \label{condition}
\end{align}
\end{lemma}
\begin{proof}
The radii drawn at the point of intersection and the line joining the centres form a right angled triangle. From Pythagorean theorem,
\begin{align}
    r_1^2 + r_2^2 = d^2
\end{align}
where $r_1$, $r_2$ are the radii and $d$ is the distance between the centre. They are given as
\begin{align}
    r_1 = \sqrt{\norm{\vec{c_1}}^2 - f_1}\\
    r_2 = \sqrt{\norm{\vec{c_2}}^2 - f_2}\\
    d = \norm{\vec{c_1} - \vec{c_2}}
\end{align}
Substituting the values
\begin{multline}
    \norm{\vec{c_1}}^2 - f_1 + \norm{\vec{c_2}}^2 - f_2 = \norm{\vec{c_1} - \vec{c_2}}^2\\
    = \norm{\vec{c_1}}^2 + \norm{\vec{c_2}}^2 - 2\vec{c_1}^\top\vec{c_2}
\end{multline}
\begin{align}
    \implies 2\vec{c_1}^\top\vec{c_2} = f_1 + f_2
\end{align}
\end{proof}
\textbf{Solution} : Let the equation of the circle be
\begin{align}
    \vec{x}^\top\vec{x} - 2\vec{c}^\top\vec{x} + f = 0
\end{align}
It passes through the point $\myvec{-3\\2}$, substituting it
\begin{align}
    \myvec{-6&4}\vec{c} - f = 13  
\end{align}
It is also orthogonal to the circles \eqref{circle1} and \eqref{circle2}
\begin{align}
    \myvec{2&2}\vec{c} - f = 1\\
    \myvec{3&-6}\vec{c} - f = -2
\end{align}
Expressing in the form of a matrix
\begin{align}
    \myvec{2&2&-1\\3&-6&-1\\-6&4&-1}\myvec{\vec{c}\\f} = \myvec{1\\-2\\13}
\end{align}
Row reducing the augumented matrix,
\begin{align}
    \myvec{2&2&-1&1\\3&-6&-1&-2\\-6&4&-1&13} \\ \xleftrightarrow[\text{$R_3$}\rightarrow{\text{$R_3 + 3R_1$}}] {\text{$R_2$}\rightarrow{\text{$2R_2 - 3R_1$}}} \myvec{2&2&-1&1\\0&-18&1&-7\\0&10&-4&16}\\
    \xleftrightarrow[\text{$R_1$}\rightarrow{\text{$2R_1 - \frac{R_3}{2}$}} \text{$,R_3$}\rightarrow{\text{$\frac{R_3}{2}$}} ] {\text{$R_2$}\rightarrow{\text{$2R_2 + \frac{R_3}{2}$}}} \myvec{4&-1&0&-6\\0&-31&0&-6\\0&5&-2&8}\\
    \xleftrightarrow[\text{$R_1$}\rightarrow{\text{$31R_1 - R_2$}}] {\text{$R_3$}\rightarrow{\text{$31R_3 + 5R_2$}}} \myvec{124&0&0&-180\\0&-31&0&-6\\0&0&-62&218}
\end{align}
\begin{align}
    \vec{c} = \myvec{\frac{-45}{31}\\\frac{6}{31}}\\
    f = \frac{-109}{31}
\end{align}
The required equation of circle,
\begin{align}
    S = \vec{x}^\top\vec{x} - 2\myvec{\frac{-45}{31}&\frac{6}{31}}\vec{x} - \frac{109}{31} = 0
\end{align}
\begin{figure}[h!]
\centering
\includegraphics[width=\columnwidth]{circles.png}
\label{fig:circles plot}
\caption{Plot of circles}
\end{figure}
\end{document}
\end{document}